% ================================================================
% LaTeX file with prefered layout for H1 paper drafts
% use: dvips -D600 file-name
% ================================================================
%%%%%%%%%%%%% LATEX HEADER
%%%%%%%%%%%%% DO NOT DELET NOT CHANGE ANYTHING IN THE HEADER
%%%%%%%%%%%%% UNLESS CLEARLY RECOMMENDED

\documentclass[12pt]{article}
%\usepackage[T1]{fontenc}
\usepackage[utf8]{inputenc}
%\bibliographystyle{desy19-080}
\bibliographystyle{utphys}
 %Choose a bibliograhpic style
\usepackage{amsmath}
\usepackage{amssymb}
%\usepackage{times}
\usepackage{txfonts}
\usepackage[mathlines,displaymath]{lineno}
\DeclareMathAlphabet{\mathbold}{OML}{txr}{b}{it}
%\usepackage[font={small,it}]{caption}
\usepackage[font={small}]{caption}

\usepackage{multirow}
%\usepackage{units}
%\usepackage{hhline}
\usepackage{dsfont}
\usepackage{xcolor}
%\usepackage{subcaption}
%\usepackage{pdflscape}
\usepackage{graphicx}
\usepackage{rotating}
\usepackage{paralist} % compactitem

%\usepackage{longtable}
\usepackage{xspace}
%\usepackage{bm} % 'bold' math symbols (better use \usepackage{newtxtext,newtxmath}, if available on latex distribution)
\usepackage{acronym}
\usepackage{dcolumn}
\newcolumntype{L}{D{.}{.}{2,5}}
%\usepackage{bbold}

%%%%%%%%%%%%% Comment the next two lines to remove the line numbering
%\usepackage[]{lineno}
%\linenumbers
%%%%%%%%%%%%%%


\usepackage{hyperref} % has to be last package loaded
\hypersetup{colorlinks=true, urlcolor=blue}
%\usepackage{cite} % Use this package to display references as [21-25], which in contrast removes the hyperlink
\usepackage{cite} % DB: enable also clickable references (must be loaded after hyperref)
\hypersetup{
  colorlinks,
  citecolor=blue,
  linkcolor=red,
  urlcolor=blue
  }
  
%%%%%%%%%%%%
\renewcommand{\topfraction}{1.0}
\renewcommand{\bottomfraction}{1.0}
\renewcommand{\textfraction}{0.0}
\renewcommand{\arraystretch}{1.25} % make lines a bit larger for tables
\newlength{\dinwidth}
\newlength{\dinmargin}
\setlength{\dinwidth}{21.0cm}
\textheight23.5cm \textwidth16.0cm
\setlength{\dinmargin}{\dinwidth}
\setlength{\unitlength}{1mm}
\addtolength{\dinmargin}{-\textwidth}
\setlength{\dinmargin}{0.5\dinmargin}
\oddsidemargin -1.0in
\addtolength{\oddsidemargin}{\dinmargin}
\setlength{\evensidemargin}{\oddsidemargin}
\setlength{\marginparwidth}{0.9\dinmargin}
\marginparsep 8pt \marginparpush 5pt
\topmargin -42pt
\headheight 12pt
\headsep 30pt \footskip 24pt
\parskip 3mm plus 2mm minus 2mm

% do not indent first line of paragraph!
%\setlength{\parindent}{0pt}
\usepackage{parskip}
\hyphenation{pa-ra-me-ters}


%%%%%%%%%%%%%%%% END OF LATEX HEADER
%===============================title page=============================
\begin{document}  

%% uncertainties: Absolute and relative uncertainties
\newcommand{\delrel}{\ensuremath{\delta}}
\newcommand{\delabs}{\ensuremath{\Delta}}
\newcommand{\dabs}[2][1]{\ensuremath{\delabs^{{\rm{#1}}}_{{#2}}}}
\newcommand{\drel}[2][1]{\ensuremath{\delrel^{{\rm{#1}}}_{{#2}}}}
\newcommand{\Nsys}{{N_{\rm sys}}}
\newcommand{\stat}{{\rm stat}}
\newcommand{\sys}{{\rm sys}}
\newcommand{\TODO}{\color{red}TODO\xspace}
\newcommand{\AS}{\color{red} AS:\xspace}

\newcommand{\muf}{\ensuremath{\mu_{f}}\xspace}
\newcommand{\mur}{\ensuremath{\mu_{r}}\xspace}
\newcommand{\as}{\ensuremath{\alpha_s}\xspace}
\newcommand{\asmz}{\ensuremath{\alpha_s(M_Z)}\xspace}
\newcommand{\asmur}{\ensuremath{\alpha_s(\mur)}\xspace}
\newcommand{\aem}{\ensuremath{\alpha_{\mathrm{em}}}\xspace}
\newcommand{\Lumi}{\ensuremath{\mathcal{L}}}
\newcommand{\pb}{\rm pb}
\newcommand{\invpb}{\ensuremath{\rm{pb}^{-1}}}
\newcommand{\PDF}{\ensuremath{{\rm PDF}}\xspace}
\renewcommand{\deg}{\ensuremath{^\circ}\xspace}
\newcommand{\unitmatrix}{1\!\!1}
\newcommand{\fC}{\ensuremath{f^{\rm C}}\xspace}
\newcommand{\fU}{\ensuremath{f^{\rm U}}\xspace}
% bold faces: no ensuremath for them
\newcommand{\bas}{\boldsymbol{{\alpha_s}}} 
\newcommand{\basmz}{\boldsymbol{\alpha_s(M_Z)}} 
\newcommand{\bmur}{\boldsymbol{\mu_{r}}}

%% unfolding
\newcommand{\chisq}{\ensuremath{\chi^{2}}}
\newcommand{\chisqA}{\ensuremath{\chi_{\rm A}^{2}}}
\newcommand{\chisqL}{\ensuremath{\chi_{\rm L}^{2}}}
\newcommand{\ndf}{\ensuremath{n_{\rm dof}}}
\newcommand{\A}{\ensuremath{\bm{A}}}
\newcommand{\M}{\ensuremath{\bm{M}}}
\newcommand{\V}{\ensuremath{\bm{V}}}
\newcommand{\B}{\ensuremath{\bm{B}}}
\newcommand{\J}{\ensuremath{\bm{J}}}
\newcommand{\N}{\ensuremath{\bm{N}}}
\newcommand{\LL}{\ensuremath{\bm{L}}}

%% calibration and cluster separation
\newcommand{\femjet}{\ensuremath{f_{\mathrm{em,jet}}}\xspace}
\newcommand{\femjetgen}{\ensuremath{f_{\mathrm{em,jet}}^{\mathrm{gen}}}\xspace}
\newcommand{\femjetrec}{\ensuremath{f_{\mathrm{em,jet}}^{\mathrm{rec}}}\xspace}
\newcommand{\Pem}{\ensuremath{P_{\mathrm{em}}}\xspace}
\newcommand{\Eem}{\ensuremath{E_{\mathrm{em}}}\xspace}
\newcommand{\Ehad}{\ensuremath{E_{\mathrm{had}}}\xspace}
%\newcommand{\Ptbal}{\ensuremath{P_{\mathrm{T}}\text{--}\mathrm{balance}}\xspace}
\newcommand{\Ptbal}{\ensuremath{P_{\mathrm{T}}}--balance\ }
\newcommand{\PTbal}{\ensuremath{P_{\mathrm{T,bal}}}\xspace}
\newcommand{\Ptgen}{\ensuremath{P_{\mathrm{T}}^\mathrm{gen}}\xspace}
\newcommand{\Ptda}{\ensuremath{P_{\mathrm{T}}^{\mathrm{da}}}\xspace}
\newcommand{\thetajet}{\ensuremath{\theta_{\mathrm{jet}}}\xspace}
\newcommand{\etajet}{\ensuremath{\eta_{\mathrm{jet}}}\xspace}
\newcommand{\Ejet}{\ensuremath{E_{\mathrm{jet}}}\xspace}
\newcommand{\Ejetda}{\ensuremath{E^{\mathrm{da}}_{\mathrm{jet}}}\xspace}
\newcommand{\relres}{\ensuremath{\sigma(\Ejet)/\Ejet}\xspace}

% hadronic variables
\newcommand{\Empz}{\ensuremath{E-p_z}\xspace}
\newcommand{\gammah}{\ensuremath{\gamma_{\mathrm{h}}}\xspace}
\newcommand{\Pth}{\ensuremath{P_{\mathrm{T}}^{\mathrm{h}}}\xspace}
\newcommand{\Pzh}{\ensuremath{p_z^{\mathrm{h}}}\xspace}
\newcommand{\h}{\ensuremath{\mathrm{h}}}

% electron variables
\newcommand{\e}{\ensuremath{\mathrm{e}}}
\newcommand{\Ee}{\ensuremath{E_\mathrm{e}^\prime}\xspace}
\newcommand{\Pte}{\ensuremath{P_{\mathrm{T}}^{\mathrm{e}}}\xspace}
\newcommand{\thetae}{\ensuremath{\theta_\mathrm{e^\prime}}\xspace}
\newcommand{\phie}{\ensuremath{\phi_\mathrm{e}}\xspace}
\newcommand{\Eda}{\ensuremath{E^{\mathrm{da}}}\xspace}

% beams
\newcommand{\Ebeam}{\ensuremath{E_{\mathrm{e}}}\xspace}
\newcommand{\Pbeam}{\ensuremath{E_{\mathrm{p}}}\xspace}

%% observables
\newcommand{\xbj}{\ensuremath{x}\xspace}
\newcommand{\Qsq}{\ensuremath{Q^2}\xspace}

% ep scattering
\newcommand{\ep}{\ensuremath{e^+}\xspace}
\newcommand{\emm}{\ensuremath{e^-}\xspace}
\newcommand{\epm}{\ensuremath{e^\pm}\xspace}

%% p_T:
\newcommand{\ptjet}{\ensuremath{P_{\rm T}^{\rm jet}}\xspace}
\newcommand{\ptjetmin}{\ensuremath{P_{\rm T,min}^{\rm jet}}\xspace}
\newcommand{\meanpt}{\ensuremath{\langle P_{\rm T} \rangle}}
\newcommand{\meanptdi}{\ensuremath{\langle P_{\mathrm{T}} \rangle_{2}}\xspace}
\newcommand{\meanpttri}{\ensuremath{\langle P_{\mathrm{T}} \rangle_{3}}\xspace}
\newcommand{\pt}{\ensuremath{P_{\rm T}}\xspace}
\newcommand{\Pt}{\ensuremath{P_{\mathrm{T}}}\xspace}
\newcommand{\Ptone}{\ensuremath{P_{\mathrm{T,1}}}\xspace}
\newcommand{\Pttwo}{\ensuremath{P_{\mathrm{T,2}}}\xspace}

\newcommand{\kt}{\ensuremath{{k_{\mathrm{T}}}}\xspace}
\newcommand{\antikt}{\ensuremath{{\mathrm{anti-}k_{\mathrm{T}}}}\xspace}
\newcommand{\bkt}{\ensuremath{\bm{k}_{\boldsymbol{\mathrm{T}}} }\xspace}
\newcommand{\bantikt}{\ensuremath{\boldsymbol{\mathrm{anti-}}\bm{k}_{\boldsymbol{\mathrm{T}}}}\xspace}

\newcommand{\etal}{{\it et al.}}
\newcommand{\Hone}{[H1 Collaboration]}


\newcommand{\Et}{\ensuremath{E_\mathrm{T}}\xspace}
\newcommand{\etalab}{\ensuremath{\eta^{\mathrm{jet}}_{\mathrm{lab}}}\xspace}
\newcommand{\ptlab}{\ensuremath{P^{\mathrm{jet}}_{\mathrm{T,lab}}}\xspace}
\newcommand{\Mjj}{\ensuremath{M_{\mathrm{12}}}\xspace}
\newcommand{\Mjjj}{\ensuremath{M_{\rm 123}}}
\newcommand{\xij}{\ensuremath{\xi}\xspace}
\newcommand{\xidi}{\ensuremath{\xi_2}\xspace}
\newcommand{\xitri}{\ensuremath{\xi_3}\xspace}

\newcommand{\ud}{\ensuremath{\mathrm{d}}\xspace}
\newcommand{\LO}{\ensuremath{\mathcal{O}(\alpha_s^0)}\xspave}
\newcommand{\Oa}{\ensuremath{\mathcal{O}(\alpha_s)}\xspace}
\newcommand{\Oaa}{\ensuremath{\mathcal{O}(\alpha_s^2)}\xspace}
\newcommand{\Oaaa}{\ensuremath{\mathcal{O}(\alpha_s^3)}\xspace}

\newcommand{\eq}{equation}
\newcommand{\fig}{figure}
\newcommand{\tab}{table}

\newcommand{\MeV}{\ensuremath{\mathrm{MeV}}\xspace}
\newcommand{\GeV}{\ensuremath{\mathrm{GeV}}\xspace}
\newcommand{\GeVsq}{\ensuremath{\mathrm{GeV}^2}\xspace}
\newcommand{\sw}{\ensuremath{{\rm sin}^2\theta_W}}
\newcommand{\dr}{\ensuremath{\Delta r}}
\newcommand{\gf}{\ensuremath{G_{\rm F}}}

\newcommand{\rhopW}[2][]{\ensuremath{\rho^{\prime}_{\text{CC}#2}}}
\newcommand{\rhop}[2][] {\ensuremath{\rho^{\prime}_{\text{NC}#2}}}
\newcommand{\kapp}[2][] {\ensuremath{\kappa^{\prime}_{\text{NC}#2}}}
\newcommand{\rhopu}{\rhop{, u}}
\newcommand{\kappu}{\kapp{, u}}
\newcommand{\rhopd}{\rhop{, d}}
\newcommand{\kappd}{\kapp{, d}}
\newcommand{\rhope}{\rhop{, e}}
\newcommand{\kappe}{\kapp{, e}}
\newcommand{\kapz}{\ensuremath{\kappa_{\text{NC}, f}}}
\newcommand{\rhoz}{\ensuremath{\rho_{\text{NC},f}}}
\newcommand{\Itf}{\ensuremath{I^3_{{\rm L},f}}}
\newcommand{\Itq}{\ensuremath{I^3_{{\rm L},q}}}

\newcommand{\fifteen}{\ensuremath{15\,000}}
\newcommand{\dd}{\mathrm{d}}
\newcommand{\Ord}{\ensuremath{\mathcal{O}}}

% couplings
\newcommand{\ad} {\ensuremath{g_A^d}}
\newcommand{\vd} {\ensuremath{g_V^d}}
\newcommand{\au} {\ensuremath{g_A^u}}
\newcommand{\vu} {\ensuremath{g_V^u}}
\newcommand{\aq} {\ensuremath{g_A^q}}
\newcommand{\vq} {\ensuremath{g_V^q}}
\newcommand{\gae}{\ensuremath{g_A^e}}
\newcommand{\ve} {\ensuremath{g_V^e}}


%% masses
\newcommand{\mt} {\ensuremath{m_t}}
\newcommand{\mW} {\ensuremath{m_W}}
\newcommand{\mWprop} {\ensuremath{m^{\rm prop}_W}}
\newcommand{\mWGfW} {\ensuremath{m^{(\gf,\mW)}_W}}
\newcommand{\mZ} {\ensuremath{m_Z}}
\newcommand{\mH} {\ensuremath{m_H}}


% Journal macro
\def\Journal#1#2#3#4{{#1}~{\bf #2} (#3) #4}
%
\def\NPB{Nucl. Phys.~}
\def\PRL{Phys. Rev. Lett.~}
\def\EPJC{Eur. Phys. J.~}
\def\PLB{Phys. Lett.~}
\def\NIM{Nucl. Instrum. Meth.~}
\def\PRD{Phys. Rev.~}
\def\JHEP{JHEP~}
\def\PROC{Conf. Proc.~}
\def\CPC{Comp. Phys. Commun.~}


%%%%%%%%%%%%%%%%%%%%%%%%%%%%%%%%%%%%%%% title page %%%%%%%%%%%%%%%%%%%%%%%%%%%%%%%%%%%%%%%%
\begin{titlepage}

\noindent
\begin{flushleft}
{\tt DESY 18-080    \hfill    ISSN 0418-9833} \\
%{\tt Spring 2018}                  \\
\end{flushleft}

\noindent
Date:      \ \ \ July 2019      \\
Version:   0.0 \\
Editors:   D.\ Britzger (daniel.britzger@desy.de), R.~\v{Z}leb\v{c}\'{i}k (radek.zlebcik@desy.de) \\
Referees:  L. Schoeffel (laurent.schoeffel@cea.fr), K. Cerny (karel.cerny@upol.cz)
\noindent

\vspace{2cm}
\begin{center}
\begin{Large}


{\bf Fit of diffractive parton distribution functions in NNLO to diffractive inclusive DIS data}

\vspace{2cm}

H1 Collaboration

\end{Large}
\end{center}

\vspace{2cm}

\begin{abstract}
\noindent
%
We present the infamous H1DPDF2019 fit.
%
\end{abstract}


\vspace{6cm}

\begin{center} To be submitted to a journal \end{center}

\end{titlepage}
\sloppy

\clearpage
%
%   REMOVE the table of contents 
%\clearpage
%\tableofcontents
%\clearpage

\section*{Todo}
\begin{itemize}
\item write DIS proceedings
\item write paper
\end{itemize}
\clearpage



%%%%%%%%%%%%%%%%%%%%%%%%%%%%%%%%%%%%%%% body starts here %%%%%%%%%%%%%%%%%%%%%%%%%%%%%%%%%%%%%%%%

%-----------------------------------------------------------------------
%   Introduction
%-----------------------------------------------------------------------
\section{Introduction}
%


%-----------------------------------------------------------------------
%   Introduction
%-----------------------------------------------------------------------
\section{Theoretical framework}


%-----------------------------------------------------------------------
%   Input data
%-----------------------------------------------------------------------
\section{Input data and fit ansatz}


%-----------------------------------------------------------------------
%   Results
%-----------------------------------------------------------------------
\section{Results}


%-----------------------------------------------------------------------
\begin{figure}[tb]
  \begin{center}
    %    \includegraphics[width=0.48\textwidth]{d18-080f2.pdf}
    \hskip0.01\textwidth
  \end{center}
  \caption{
    Value of the $W$-boson mass compared to results obtained by the ATLAS, ALEPH, CDF,
    D0, DELPHI, L3 and OPAL experiments, and the world average value.
    The inner error bars indicate statistical uncertainties and the
    outer error bars full uncertainties.
}
\label{fig:mW}
\end{figure}




%-----------------------------------------------------------------------
%                            Summary
%-----------------------------------------------------------------------
\section{Summary}
\label{sect:Conclusion}

Summary

%%%%%%%%%%%%%%%%%%%%%%%%%%%%%%%%%%%%%%%%%%%%%%%%%%%%%%%%%%%%%%%%%%%%%

\section*{Acknowledgements}
We are grateful to the HERA machine group whose outstanding
efforts have made this experiment possible.
We thank the engineers and technicians for their work in constructing and
maintaining the H1 detector, our funding agencies for
financial support, the
DESY technical staff for continual assistance
and the DESY directorate for support and for the
hospitality which they extend to the non--DESY
members of the collaboration.

We would like to give credit to all partners contributing to the EGI
computing infrastructure for their support for the H1 Collaboration.  

We express our thanks to all those involved in securing not only the
H1 data but also the software and working environment for long term
use, allowing the unique H1 data set to continue to be explored in the
coming years. The transfer from experiment specific to central
resources with long term support, including both storage and batch
systems, has also been crucial to this enterprise. We therefore also
acknowledge the role played by DESY-IT and all people involved during
this transition and their future role in the years to come.  


%%%%%%%%%%%%%%%%%%%%%%%%%%%%%%%%%%%%%%%%%%%%%%%%%%%%%%%%%%%%%%%

%%%%%%%%%%%%%%%%%%%%%%%%%%%%%%%%%%%%%%%%%%%%%%%%%%%%%%%%%%%%%%%%%

\clearpage
\appendix
\section*{Appendix}
\section{DPDF code}
\label{appx:DPDFcode}

% -----------------------------------------------------------------------------
%\begin{table}[ht]
%  \footnotesize
%  \centering
%   \begin{tabular}{ccr@{$\,\pm\,$}l|cccc} 
%   \hline
%   \Qsq\ range $[\GeVsq]$ & Parameter & \multicolumn{2}{c}{Result} & \multicolumn{4}{c}{Correlation} \\
%   \hline
%   $[224,708]$    & \rhopW{,f} &    0.976 &  0.018 &  1.00  \\
%   $[708,2239]$   & \rhopW{,f} &    0.998 &  0.011 &  0.47 &  1.00  \\
%   \hline
%   \end{tabular}
%   \caption{
%     Caption
%   }
%     \label{tab:example}
%\end{table}
%
%



%%======================= References ==========================%%
%%%%%%%%%%%%%%%%%%%%%%%%%%%%%%%%%%%%%%%%%%%%%%%%%%%%%%%%%%%%%%%%%

\clearpage

\begin{flushleft}
\bibliography{h1dpdf2019}
\end{flushleft}


\end{document}
